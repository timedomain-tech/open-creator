\section{Introduction}

AI agents engage in complex reasoning by integrating planning, decision-making, and the utilization of appropriate tools or APIs. However, these tools are typically predetermined and designed by humans, and the number of available tools is often limited due to the constraints on the input context length of Large Language Models (LLMs). To enhance the versatility of AI agents, a viable approach is to amalgamate the code-generation capabilities of LLMs with code execution functionalities. This integration allows for the flexible writing and execution of code to address specific user needs, embodying the role of Code Interpreters.

Given that LLMs occasionally generate erroneous codes—leading to low robustness and inability to meet user requirements—recent research has focused on enabling LLMs to auto-correct codes through environmental feedback. Additionally, there is emphasis on developing sophisticated projects through rational task decomposition and persistent memory. This focus has given rise to a plethora of AI agent frameworks, including MetaGPT, ChatDev, GPT-enginger, GPT-term, and codeplan. These studies explore collaborative mechanisms among different roles, introduction of improved environments, enhanced feedback from agents, optimized task decomposition, and various engineering tricks, collectively contributing to the flourishing ecosystem of AI agents in the fields of Computer Science and Software Engineering. A comprehensive literature review in this area has been conducted by L Wang.

% Despite their stellar performance in handling a multitude of tasks, these agents still face challenges in persisting, sharing, and updating skill libraries. These challenges become more prominent when extracting, testing, and refactoring skills from various sources. To address these issues, we introduce open-creator, a novel framework designed to assist AI agents in efficiently unifying, testing, and refactoring skills from diverse sources.

% In contrast to traditional approaches, open-creator adopts a modular strategy, allowing developers and researchers to create, share, and reuse skills seamlessly without concerns over compatibility or version control. The framework encompasses crucial functionalities such as skill extraction through user interactions, persistent skill saving, demand-based skill search and usage, and skill sharing to enhance the community's knowledge base. 

% Moreover, our AI agents offer an \textbf{Extractor agent} that transforms inputs from multiple avenues into uniform skill objects, an \textbf{Interpreter agent} creating and interpreting problem-solving experiences, an \textbf{Tester agent} ensuring robustness and generalization capabilities, an \textbf{Refactor agent} assisting in skill modification, restructuring, combination, and decomposition, and a \textbf {Creator agent} coordinating the processes of creation, saving, searching, testing, and refactoring.

% \section{Introduction}
% \lipsum[2]
% \lipsum[3]


% \section{Headings: first level}
% \label{sec:headings}

% \lipsum[4] See Section \ref{sec:headings}.

% \subsection{Headings: second level}
% \lipsum[5]
% \begin{equation}
% \xi _{ij}(t)=P(x_{t}=i,x_{t+1}=j|y,v,w;\theta)= {\frac {\alpha _{i}(t)a^{w_t}_{ij}\beta _{j}(t+1)b^{v_{t+1}}_{j}(y_{t+1})}{\sum _{i=1}^{N} \sum _{j=1}^{N} \alpha _{i}(t)a^{w_t}_{ij}\beta _{j}(t+1)b^{v_{t+1}}_{j}(y_{t+1})}}
% \end{equation}

% \subsubsection{Headings: third level}
% \lipsum[6]

% \paragraph{Paragraph}
% \lipsum[7]

% \section{Examples of citations, figures, tables, references}
% \label{sec:others}
% \lipsum[8] \cite{kour2014real,kour2014fast} and see \cite{hadash2018estimate}.

% The documentation for \verb+natbib+ may be found at
% \begin{center}
%   \url{http://mirrors.ctan.org/macros/latex/contrib/natbib/natnotes.pdf}
% \end{center}
% Of note is the command \verb+\citet+, which produces citations
% appropriate for use in inline text.  For example,
% \begin{verbatim}
%    \citet{hasselmo} investigated\dots
% \end{verbatim}
% produces
% \begin{quote}
%   Hasselmo, et al.\ (1995) investigated\dots
% \end{quote}

% \begin{center}
%   \url{https://www.ctan.org/pkg/booktabs}
% \end{center}


% \subsection{Figures}
% \lipsum[10] 
% See Figure \ref{fig:fig1}. Here is how you add footnotes. \footnote{Sample of the first footnote.}
% \lipsum[11] 

% \begin{figure}
%   \centering
%   \fbox{\rule[-.5cm]{4cm}{4cm} \rule[-.5cm]{4cm}{0cm}}
%   \caption{Sample figure caption.}
%   \label{fig:fig1}
% \end{figure}

% \subsection{Tables}
% \lipsum[12]
% See awesome Table~\ref{tab:table}.

% \begin{table}
%  \caption{Sample table title}
%   \centering
%   \begin{tabular}{lll}
%     \toprule
%     \multicolumn{2}{c}{Part}                   \\
%     \cmidrule(r){1-2}
%     Name     & Description     & Size ($\mu$m) \\
%     \midrule
%     Dendrite & Input terminal  & $\sim$100     \\
%     Axon     & Output terminal & $\sim$10      \\
%     Soma     & Cell body       & up to $10^6$  \\
%     \bottomrule
%   \end{tabular}
%   \label{tab:table}
% \end{table}

% \subsection{Lists}
% \begin{itemize}
% \item Lorem ipsum dolor sit amet
% \item consectetur adipiscing elit. 
% \item Aliquam dignissim blandit est, in dictum tortor gravida eget. In ac rutrum magna.
% \end{itemize}
